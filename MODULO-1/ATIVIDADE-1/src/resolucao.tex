\documentclass{assignment}
\begin{document}

\assignmentTitle{Matheus Rocha Monteiro}{assets/ifce.png}{Trainee Data Science - Lapisco}{Módulo 1 - Atividade 1}

\section*{SQL}

\noindent Para resolver os exercícios, utilize a descrição do modelo relacional abaixo e os comandos SQL
aprendidos. Se houver a necessidade de utilizar mais de uma tabela em uma consulta, utilize o
INNER JOIN, em vez do produto cartesiano.

\subsection*{Questão 1}
Insira 3 registros para cada tabela criada.
\subsubsection*{Solução}
\lstinputlisting[language=sql]{../code/q1.sql}

\subsection*{Questão 2}
Encontre todas as novelas que tenham o valor do horário de exibição vazio.
\subsubsection*{Solução}
\lstinputlisting[language=sql]{../code/q2.sql}

\subsection*{Questão 3}
Selecione o nome de todos os atores que morem em cidades que comecem com a letra “M”.
\subsubsection*{Solução}
\lstinputlisting[language=sql]{../code/q3.sql}

\subsection*{Questão 4}
Selecione todos os campos da tabela tbPersonagem ordenados por nome em ordem crescente.
\subsubsection*{Solução}
\lstinputlisting[language=sql]{../code/q4.sql}

\subsection*{Questão 5}
Selecione quantos capítulos existem por novela, retorne o nome da novela e a quantidade de
capítulos para a novela.
\subsubsection*{Solução}
\lstinputlisting[language=sql]{../code/q5.sql}

\subsection*{Questão 6}
Encontre o nome de todas as novelas que tem mais de 40 capítulos.
\subsubsection*{Solução}
\lstinputlisting[language=sql]{../code/q6.sql}

\section*{Pandas}

Nas questões seguintes, utilizar as seguintes bases de dados: titanic, iris.

\subsection*{Questão 1}
Verifique a presença de outliers (e.g., valor maior que 3 vezes o desvio padrão da base) na base
de dados iris ou outra base de dados. Caso não exista, insira uma ou mais linhas e selecione elas
utilizando a regra mencionada.
\subsubsection*{Solução}
\lstinputlisting[language=python,inputencoding=utf8]{../code/q1.py}

\subsection*{Questão 2}
Faça um merge de dois dataframes e crie uma nova coluna para indicar a que dataframe pertence
cada linha.
\subsubsection*{Solução}
\lstinputlisting[language=python,inputencoding=utf8]{../code/q2.py}
\subsubsection*{Resultado}
É gerado um dataframe com os dados de ambos dataframes anteriores e também com um campo adicional, contendo a informação de onde tal informação veio:
\begin{table}[H]
    \centering
    \caption{Dataset gerado através da concatenação dos datasets 1 e 2}
    \begin{tabular}{lrrl}
        \toprule
        {} &  a &   b &    data\_from \\
        \midrule
        0 &  1 &  10 &  dataframe 1 \\
        1 &  2 &  11 &  dataframe 1 \\
        2 &  3 &  12 &  dataframe 1 \\
        3 &  4 &  13 &  dataframe 2 \\
        4 &  5 &  14 &  dataframe 2 \\
        5 &  6 &  15 &  dataframe 2 \\
        \bottomrule
    \end{tabular}
\end{table}

\subsection*{Questão 3}
Selecione e mostre a quantidade de passageiros do titanic que possuem mais de 27 anos e que
sobreviveram ao acidente.
\subsubsection*{Solução}
\lstinputlisting[language=python,inputencoding=utf8]{../code/q3.py}
\subsubsection*{Resultado}
\begin{table}[H]
    \centering
    \caption{5 primeiros passageiros que possuem mais de 27 anos e sobreviveram ao acidente}

    \resizebox{\textwidth}{!}{\begin{tabular}{llrlrrrrllllllll}
        \toprule
        {} &  survived &  pclass &     sex &   age &  sibsp &  parch &     fare & embarked &   class &    who &  adult\_male & deck &  embark\_town & alive &  alone \\
        \midrule
        1  &      True &       1 &  female &  38.0 &      1 &      0 &  71.2833 &        C &   First &  woman &       False &    C &    Cherbourg &   yes &  False \\
        3  &      True &       1 &  female &  35.0 &      1 &      0 &  53.1000 &        S &   First &  woman &       False &    C &  Southampton &   yes &  False \\
        11 &      True &       1 &  female &  58.0 &      0 &      0 &  26.5500 &        S &   First &  woman &       False &    C &  Southampton &   yes &   True \\
        15 &      True &       2 &  female &  55.0 &      0 &      0 &  16.0000 &        S &  Second &  woman &       False &  NaN &  Southampton &   yes &   True \\
        21 &      True &       2 &    male &  34.0 &      0 &      0 &  13.0000 &        S &  Second &    man &        True &    D &  Southampton &   yes &   True \\
        \bottomrule
    \end{tabular}}
\end{table}

\subsection*{Questão 4}
Transforme os dados contínuos de idade do titanic em dados categóricos de acordo com a
seguinte regra:

\begin{itemize}
    \item 0 a 18 anos $\rightarrow$ Criança
    \item 18 a 65 anos $\rightarrow$ Adulto
    \item maior de 65 $\rightarrow$ Idoso
\end{itemize}

\subsubsection*{Solução}
\lstinputlisting[language=python,inputencoding=utf8]{../code/q4.py}
\subsubsection*{Resultado}
\begin{table}[H]
    \centering
    \caption{Amostra de 10 passageiros após classificação baseada em idade}

    \resizebox{\textwidth}{!}{\begin{tabular}{llrlrrrrlllllllll}
        \toprule
        {} &  survived &  pclass &     sex &   age &  sibsp &  parch &     fare & embarked &   class &    who &  adult\_male & deck &  embark\_town & alive &  alone & person\_type \\
        \midrule
        862 &      True &       1 &  female &  48.0 &      0 &      0 &  25.9292 &        S &   First &  woman &       False &    D &  Southampton &   yes &   True &       Adult \\
        223 &     False &       3 &    male &   NaN &      0 &      0 &   7.8958 &        S &   Third &    man &        True &  NaN &  Southampton &    no &   True &         Old \\
        84  &      True &       2 &  female &  17.0 &      0 &      0 &  10.5000 &        S &  Second &  woman &       False &  NaN &  Southampton &   yes &   True &       Child \\
        680 &     False &       3 &  female &   NaN &      0 &      0 &   8.1375 &        Q &   Third &  woman &       False &  NaN &   Queenstown &    no &   True &         Old \\
        535 &      True &       2 &  female &   7.0 &      0 &      2 &  26.2500 &        S &  Second &  child &       False &  NaN &  Southampton &   yes &  False &       Child \\
        623 &     False &       3 &    male &  21.0 &      0 &      0 &   7.8542 &        S &   Third &    man &        True &  NaN &  Southampton &    no &   True &       Adult \\
        148 &     False &       2 &    male &  36.5 &      0 &      2 &  26.0000 &        S &  Second &    man &        True &    F &  Southampton &    no &  False &       Adult \\
        3   &      True &       1 &  female &  35.0 &      1 &      0 &  53.1000 &        S &   First &  woman &       False &    C &  Southampton &   yes &  False &       Adult \\
        34  &     False &       1 &    male &  28.0 &      1 &      0 &  82.1708 &        C &   First &    man &        True &  NaN &    Cherbourg &    no &  False &       Adult \\
        241 &      True &       3 &  female &   NaN &      1 &      0 &  15.5000 &        Q &   Third &  woman &       False &  NaN &   Queenstown &   yes &  False &         Old \\
        \bottomrule
        \end{tabular}}
\end{table}


\subsection*{Questão 5}
Aplique o hold out utilizando diferentes porcentagens de divisão para separar a base iris em
treino e teste (e.g. 60/40, 70/30, 80/20, 90/10). Crie vetores para armazenar os dados da divisão de
treino e teste da seguinte forma:

\begin{itemize}
    \item $X\_train$ $\rightarrow$ armazena os dados de treino
    \item $X\_test$ $\rightarrow$ armazena os dados de teste
    \item $y\_train$ $\rightarrow$ armazena a label correspondente dos dados de treino
    \item $y\_test$ $\rightarrow$ armazena a label correspondente dos dados de teste
\end{itemize}

\subsubsection*{Solução}
\lstinputlisting[language=python,inputencoding=utf8]{../code/q5.py}
\end{document}