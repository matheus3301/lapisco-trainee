\documentclass{assignment}
\begin{document}

\assignmentTitle{Matheus Rocha Monteiro}{assets/ifce.png}{Trainee Data Science - Lapisco}{Módulo 1 - Atividade 1}

\section*{SQL}

\noindent Para resolver os exercícios, utilize a descrição do modelo relacional abaixo e os comandos SQL
aprendidos. Se houver a necessidade de utilizar mais de uma tabela em uma consulta, utilize o
INNER JOIN, em vez do produto cartesiano.

\subsection*{Questão 1}
Insira 3 registros para cada tabela criada.
\subsubsection*{Solução}
\lstinputlisting[language=sql]{../code/q1.sql}

\subsection*{Questão 2}
Encontre todas as novelas que tenham o valor do horário de exibição vazio.
\subsubsection*{Solução}
\lstinputlisting[language=sql]{../code/q2.sql}

\subsection*{Questão 3}
Selecione o nome de todos os atores que morem em cidades que comecem com a letra “M”.
\subsubsection*{Solução}
\lstinputlisting[language=sql]{../code/q3.sql}

\subsection*{Questão 4}
Selecione todos os campos da tabela tbPersonagem ordenados por nome em ordem crescente.
\subsubsection*{Solução}
\lstinputlisting[language=sql]{../code/q4.sql}

\subsection*{Questão 5}
Selecione quantos capítulos existem por novela, retorne o nome da novela e a quantidade de
capítulos para a novela.
\subsubsection*{Solução}
\lstinputlisting[language=sql]{../code/q5.sql}

\subsection*{Questão 6}
Encontre o nome de todas as novelas que tem mais de 40 capítulos.
\subsubsection*{Solução}
\lstinputlisting[language=sql]{../code/q6.sql}

\end{document}